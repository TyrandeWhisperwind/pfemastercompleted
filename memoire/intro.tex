\begin{center}
	\textbf{\LARGE Introduction générale}
\end{center}
\subsubsection*{}
Avec le développement des technologies de l’information et de la communication et de l’internet, les moyens et les possibilités d’échanges et de partage de données ont largement évoluées. Cependant, le volume de données créé chaque jour est en croissance exponentiel. Devant cette masse importante de données, plusieurs problèmes sont engendrés dont la difficulté à retrouver la bonne information adéquate selon les besoins et préférences des utilisateurs. Par conséquent, l’utilisation des systèmes de recommandation devient une nécessité et un moyen incontournable pour avoir les ressources les plus appropriés à un utilisateur donné. Les systèmes de recommandation sont en expansion continue depuis les années 1990. Ils offrent aux utilisateurs la possibilité d’accéder aux contenus qui leur sont les plus adaptés et appropriés parmi les différents contenus disponibles, de manière personnalisée selon leurs profils leurs préférences et centres d’intérêts…. Depuis leur apparition, les systèmes de recommandation ont beaucoup évolué, de la définition d’approches de recommandation simples à la mise en place d’approches hybrides qui combinent plusieurs algorithmes de recommandation afin de gagner en performance et de pallier les limitations des algorithmes standards. Cependant, les systèmes de recommandation ont encore besoin d’être améliorés afin d’augmenter la satisfaction des utilisateurs. Nous pouvons trouver actuellement des systèmes combinant les systèmes classiques avec d’autres informations telles que l’information sémantique ou encore sociale dans les réseaux sociaux par exemple.\\

Dans ce même contexte, nous nous intéressons dans le cadre de ce projet de fin d’études de master, à la proposition d’une approche de recommandation de contenu en considérant la représentation sémantique. Dans un premier temps, nous nous sommes basées sur la combinaison du filtrage collaboratif (FC), l’un des algorithmes les plus utilisés, avec le filtrage sémantique (FSem), en proposant trois hybridations différentes. Nous avons considéré dans l’hybridation différentes variantes du FC (FC basé utilisateur, FC basé item, FC avec SVD…). Puis, dans un second temps, nous avons appliqué deux techniques de classification sur le FC et FSem : classification supervisée (K-NN - K plus proches voisins) et non supervisée (Kmedoids). L’hybridation dans ce cas a été appliquée sur les algorithmes avec classification. Finalement, afin d’améliorer les performances de l’algorithme Kmedoids, nous avons utilisé la méta-heuristique de colonies d’abeilles (BSO - Bees Swarm Optimization Algorithm).\\

En se basant sur l’approche proposée qui inclut plusieurs algorithmes FC standard, FC avec classification et optimisation, FSem standard, FSem avec classification et optimisation, filtrage hybride (FHyb) et FHhyb avec classification et optimisation, nous avons implémenté un prototype de système de recommandation implémentant tous ces algorithmes. Sachant que notre système a été implémenté de manière flexible, permettant ainsi une évaluation selon le choix de l’utilisateur. A titre d’exemple, l’utilisateur peut choisir la variante souhaitée du FC et du FSem et construire sur la base de ce choix son hybridation selon les trois hybridations offertes. Il aura aussi la possibilité d’appliquer une classification avec ou sans optimisation, de choisir une hybridation multi-vues ou non.\\

Après cette introduction, notre mémoire sera structuré comme suit :
\begin{itemize}
\item Chapitre1 : présente notre état de l’art en décrivant quelques notions liées au recommandation.
\item Chapitre2 : présente une étude sur les méthodes de classification et d’optimisation.
\item Chapitre 3 : décrit en détail notre approche de recommandation basée sur les algorithmes de filtrage collaboratif et sémantique et les techniques de classification et d’optimisation.
\item Chapitre 4 : présente les résultats  des évaluations  des différents algorithmes implémentés en utilisant deux bases de données : la base connue MovieLens-100K et la base Epinions enrichie par l’aspect sémantique (RED- Rich Epinions Dataset).

\end{itemize}
Finalement, nous terminerons par une conclusion générale et quelques perspectives.
